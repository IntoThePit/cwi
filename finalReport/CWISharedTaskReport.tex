%
% File acl2018.tex
%
%% Based on the style files for ACL-2017, with some changes, which were, in turn,
%% Based on the style files for ACL-2015, with some improvements
%%  taken from the NAACL-2016 style
%% Based on the style files for ACL-2014, which were, in turn,
%% based on ACL-2013, ACL-2012, ACL-2011, ACL-2010, ACL-IJCNLP-2009,
%% EACL-2009, IJCNLP-2008...
%% Based on the style files for EACL 2006 by 
%%e.agirre@ehu.es or Sergi.Balari@uab.es
%% and that of ACL 08 by Joakim Nivre and Noah Smith

\documentclass[11pt,a4paper]{article}
\usepackage[hyperref]{acl2018}
\usepackage{times}
\usepackage{latexsym}

\usepackage{csquotes}

\usepackage{url}

\aclfinalcopy % Uncomment this line for the final submission
%\def\aclpaperid{***} %  Enter the acl Paper ID here

%\setlength\titlebox{5cm}
% You can expand the titlebox if you need extra space
% to show all the authors. Please do not make the titlebox
% smaller than 5cm (the original size); we will check this
% in the camera-ready version and ask you to change it back.

\newcommand\BibTeX{B{\sc ib}\TeX}

\title{Class project: Complex Word Identification}

\author{Pierre Finnimore}

\date{}

\begin{document}
\maketitle
\begin{abstract}

\end{abstract}

\section{Introduction}

Initial words: 271

What is the task and why is it important?

The task is as follows: given a target word (or set of words) within a sentence, identify if the target is \enquote{complex}. The data given for this task is a set of labelled sentences with targets. The labels were derived from a survey of both Native and Non-native speakers of two languages: English and Spanish.

We are interested in identifying word complexity for several reasons. Automatic extraction of complex terms could help with automated tutoring systems, Natural Language Generation, writing editing software, studies into second-language acquisition, political speech analysis, Machine Translation, as well as linguistic or psychological studies into the what people find complex. 

\section{Baseline system description}

System descriptions in enough detail for the reader to be able to understand how to reimplement your baseline models and to appreciate why they are suitable for the task at hand.

\section{Improved system motivation and description}



\section{Experiments on development set}

Does your idea work as expected? Evaluate on the test set the baseline and the improved system, is it still the case? Identify examples in development data which help showcase why the improved system works better.

\section{Learning curves}

Plot learning curves for the trainable systems you experiment with. Are some systems better than others when less training data is available?

\section{Examples of failed predictions}

Identify examples where your improved system fails to predict correctly and propose ideas for future work to address them.

\section{Conclusions}

what have we learnt from your experiments that could inform future work


% include your own bib file like this:
%\bibliographystyle{acl}
%\bibliography{acl2018}

%\bibliography{mybib}
%\bibliographystyle{acl_natbib}

\end{document}
